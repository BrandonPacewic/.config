\setcounter{tocdepth}{4}

\usepackage{hyperref}
\hypersetup{
    colorlinks=true,
    linkcolor=blue,
}

\usepackage[usenames,dvipsnames]{xcolor}
\newcommand{\Black}[1]{\textcolor{black}{#1}}

\usepackage{amsmath}
\usepackage{enumerate}
\usepackage[margin=0.8in]{geometry}
\usepackage{tikz}
\usepackage{subfiles}
\usepackage{indentfirst}

\usepackage{pgfplots}
\pgfplotsset{compat=1.18}

\usepackage{subcaption}
\usepackage{fancyhdr}

\usepackage{graphicx}
\graphicspath{ {./images/} }

\usepackage{multicol}

\usepackage{titlesec}
\titleformat{\chapter}[display]
    {\normalfont\bfseries}{}{0pt}{\Huge}
\titlespacing\subsubsection{0pt}{0pt}{0pt}

\usepackage{amsthm}
\usepackage{thmtools}
\usepackage[framemethod=TikZ]{mdframed}

\declaretheoremstyle[
    headfont=\bfseries\sffamily\color{ForestGreen!70!black},
    headpunct={\newline},
    bodyfont=\normalfont,
    mdframed={
        linewidth=2pt,
        rightline=false, topline=false, bottomline=false,
        linecolor=ForestGreen, backgroundcolor=ForestGreen!5,
    }
]{thmgreenline}

\declaretheoremstyle[
    headfont=\bfseries\sffamily\color{NavyBlue!70!black},
    headpunct={\newline},
    bodyfont=\normalfont,
    mdframed={
        linewidth=2pt,
        rightline=false, topline=false, bottomline=false,
        linecolor=NavyBlue, backgroundcolor=NavyBlue!5,
    }
]{thmblueline}

\declaretheoremstyle[
    headfont=\bfseries\sffamily\color{NavyBlue!70!black},
    headpunct={\newline},
    bodyfont=\normalfont,
    mdframed={
        linewidth=2pt,
        rightline=false, topline=false, bottomline=false,
        linecolor=NavyBlue, backgroundcolor=NavyBlue!1,
    }
]{thmproof}

\declaretheoremstyle[
    headformat={},
    headpunct={},
    bodyfont=\normalfont,
    mdframed={
        linewidth=2pt,
        rightline=false, topline=false, bottomline=false,
        linecolor=RawSienna, backgroundcolor=RawSienna!2,
    }
]{thmlecture}

\declaretheorem[style=thmgreenline, name=Definition]{definition}
\declaretheorem[style=thmblueline, name=Theorem]{theorem}
\declaretheorem[style=thmlecture]{lecture}
\declaretheorem[style=thmblueline, numbered=no, name=Note]{note}
\declaretheorem[style=thmblueline, numbered=no, name=Remark]{remark}
\declaretheorem[style=thmblueline, numbered=no, name=Example]{example}
\declaretheorem[style=thmblueline, numbered=no, name=Rule]{myrule}

\declaretheorem[style=thmproof, numbered=no, name=Proof]{replacementproof}
\renewenvironment{proof}{\vspace{-16.5pt}\replacementproof}{\endreplacementproof}

\newcommand{\Lim}{\displaystyle\lim}
\newcommand{\Sum}{\displaystyle\sum}
\newcommand{\Int}{\displaystyle\int}
\newcommand{\Example}{\noindent\textbf{Example}} % No longer used, here for backwards compatibility with older notes.
\newcommand{\Startlectures}{}
\newcommand{\Endlectures}{}

\newcommand{\Ln}{\ln}
\newcommand{\Sin}{\sin}
\newcommand{\Cos}{\cos}
\newcommand{\Tan}{\tan}
\newcommand{\Csc}{\csc}
\newcommand{\Sec}{\sec}
\newcommand{\Cot}{\cot}

\renewcommand{\Pi}{\pi}
\renewcommand{\Theta}{\theta}
\renewcommand{\Beta}{\beta}
\renewcommand{\Alpha}{\alpha}
\renewcommand{\Gamma}{\gamma}
